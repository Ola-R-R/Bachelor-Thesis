
\section{Modelling Sports Results}

\noindent Modelling sport results can have many different use cases. It can be used to predict future matches, player performance among other things. Coaches and analysts can use data to analyze strengths and weaknesses of players and opponents, and scouts can use models to find what players could have a potential future in that sport. The main objective of this thesis is to find a model that fit 8 seasons worth of data from the National Basketball Association (NBA), starting with the 2014-2015 season, skipping the 2019-2020 season, and ending with the 2022-2023 season. The reason why the 2019-2020 season is skipped is because this season got cut short because of the covid-19-pandemic. Something similar has been done in \cite{fotballglmm}, were they modelled the scores in football matches to find the attack strength and defence strength of teams, but they also found the home-field advantage that a team has. In this thesis I will also find the home-court advantage but also find the differences between the 2020-2021 season and the other seasons. Due to covid restrictions, this season was played without any fans, so it will be interesting to see if the home-court advantage disappeared or got weaker. Further differences between what they have done and what I have done in this thesis will be looked at in Section \ref{bestmod}. \\

\newpage

\section{NBA}

\noindent The NBA, National Basketball Association, is the biggest basketball league in North-America. A regular season in the NBA, consists of 82 games for each team. There are 30 teams in the NBA, divided into 2 conferences, the eastern- and the western conference. Teams in the same conference plays 4 games against each other and 2 games against teams in the other conference. The 2020 to 2021 season, reffered to as the Covid season, only consisted of 72 games for each team. Also, under this season there where no fans allowed because of Covid-19 regulations.

There are 3 different scoring methods, or types, in a normal NBA game. You have the one-pointers, scoring method one, which are scored from free throws. Free throws are a form of penalties obtained when the other team commits a foul. Two-pointers, scoring method two, are obtained from getting the basketball into the opponents hoop while inside the three point line, while three-pointers, scoring method three, are obtained from scoring outside the three point line. Also, a team is awarded 2 free throws when fouled while attempting a two pointer, and 3 free throws when fouled while attempting a three pointer. But they are only awarded 1 free throw if they scored while being fouled.

\section{Home Court Advantage}

\noindent In \cite{homecourtadv}, the author analyzed home court advantages for NCAA basketball statistics. NCAA, National Collegiate Athletic Association, is the college equivalent of the NBA. He came to the conclusion that when playing on their home court, a team received a boost in nearly all statistical categories. In this thesis, I will come to the same conclusion that teams scored more points while playing on their home court.
